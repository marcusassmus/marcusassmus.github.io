%We are running pdfLaTeX!
%
\documentclass[11pt,a4paper,fleqn]{scrartcl}
%%%%%%%%%%%%%%%%%%%%%%%%%%%%%%%%%%%%%%%%%%%%%%%%%%%%%%%%%
%
% standard packages 
%
%%%%%%%%%%%%%%%%%%%%%%%%%%%%%%%%%%%%%%%%%%%%%%%%%%%%%%%%%
\usepackage[utf8]{inputenc}
\usepackage[T1]{fontenc}
\usepackage{lmodern} 
\usepackage{float}
\usepackage{ngerman}
\usepackage[usenames,dvipsnames,svgnames,table]{xcolor}
\usepackage{multicol}
%%%%%%%%%%%%%%%%%%%%%%%%%%%%%%%%%%%%%%%%%%%%%%%%%%%%%%%%%
%
% geometry page
%
%%%%%%%%%%%%%%%%%%%%%%%%%%%%%%%%%%%%%%%%%%%%%%%%%%%%%%%%%
\usepackage{geometry}
\geometry{left=1.5cm,textwidth=17.8cm,top=1.4cm,textheight=26cm} % für Hausaufgaben, makrotypographisch wenig sinnvoll
%%%%%%%%%%%%%%%%%%%%%%%%%%%%%%%%%%%%%%%%%%%%%%%%%%%%%%%%%
%
% math packages
%
%%%%%%%%%%%%%%%%%%%%%%%%%%%%%%%%%%%%%%%%%%%%%%%%%%%%%%%%%
\usepackage{amsmath}
\usepackage{amsfonts}
\usepackage{amssymb}
\usepackage{calligra} % Font für Mengen und Gruppen
\usepackage{mathrsfs}
\everymath=\expandafter{\the\everymath\displaystyle}
\usepackage{xfrac} % beinhaltet \sfrac für schräge Brüche im Text
%%%%%%%%%%%%%%%%%%%%%%%%%%%%%%%%%%%%%%%%%%%%%%%%%%%%%%%%%
%
% own macros
%
%%%%%%%%%%%%%%%%%%%%%%%%%%%%%%%%%%%%%%%%%%%%%%%%%%%%%%%%%
\newcommand{\dil}{\mathrm{dil}} %dilatorisch
\newcommand{\dev}{\mathrm{dev}} %deviatorisch
\renewcommand{\b}{\boldsymbol} % allgemein fette Mathematik
\newcommand{\ve}{\boldsymbol} %Vektoren
\newcommand{\tz}{\boldsymbol} % Tensoren zweiter Stufe
\DeclareMathAlphabet{\mathbfsf}{\encodingdefault}{\sfdefault}{bx}{sl}
\newcommand{\tv}[1]{{\mathbfsf{#1}}} % tensoren vierter Stufe
% ein paar Operatoren
\renewcommand{\d}{\, \mathrm{d}} % Zeichen für z.B. "Ableitung ... nach ..."
\newcommand{\pd}{\, \partial} % Zeichen für z.B. "partielle Ableitung ... nach ..."
\renewcommand{\div}{\, \mathrm{div} \,} % Divergenz
\newcommand{\grad}{\, \mathrm{grad} \,} % Gradient
\newcommand{\rot}{\, \mathrm{rot} \,} % Rotation
\newcommand{\spur}{\, \mathrm{spur} \,} % Rotation
\renewcommand{\sp}{\b{\cdot}}  % Skalar-Produkt
\newcommand{\kp}{\times}  % Kreuz-Produkt
\newcommand{\dyad}{\otimes}  % Dyadisches Produkt
% Matrizeninhalt rechtsbündig
\makeatletter
\renewcommand*\env@matrix[1][*\c@MaxMatrixCols c]{%
  \hskip -\arraycolsep
  \let\@ifnextchar\new@ifnextchar
  \array{#1}}
\makeatother
% Mengen und Gruppen --- in Text und Matheumgebungen nutzbar!
\newcommand{\lin}{\,\text{\calligra{Lin}}}  
\newcommand{\inv}{\,\text{\calligra{Inv}}}  
\newcommand{\invpl}{\,\text{\calligra{Inv}$^+$}} 
\newcommand{\invmi}{\,\text{\calligra{Inv}$^-$}} 
\newcommand{\unim}{\,\text{\calligra{Unim}}}  
\newcommand{\unimpl}{\,\text{\calligra{Unim}$^+$}}  
\newcommand{\unimmi}{\,\text{\calligra{Unim}$^-$}}  
\newcommand{\orth}{\,\text{\calligra{Orth}}} 
\newcommand{\orthpl}{\,\text{\calligra{Orth}$^{\;\;+}$}} 
\newcommand{\orthmi}{\,\text{\calligra{Orth}$^{\;\;-}$}} 
\newcommand{\iso}{\,\text{\calligra{Iso}}} 
\newcommand{\isopl}{\,\text{\calligra{Iso}$^{\;+}$}} 
\newcommand{\isomi}{\,\text{\calligra{Iso}$^{\;-}$}} 
\newcommand{\sym}{\,\text{\calligra{Sym}}} 
\newcommand{\skw}{\,\text{\calligra{Skw}}} 
\newcommand{\psym}{\,\text{\calligra{Psym}}} 
% Räume --- in Text und Matheumgebungen nutzbar!
\newcommand{\R}{\;\text{\calligra{R}}}  
\newcommand{\V}{\;\text{\calligra{V}}}
\newcommand{\E}{\;\text{\calligra{E}}}
% Körper --- in Text und Matheumgebungen nutzbar!
% hier ohne den Calligra-Font, das sonst zu groß im Vergleich zu \pd
\renewcommand{\k}{\;$\mathscr{B}$}
\newcommand{\kn}{\;$\mathscr{B}_0$}
\newcommand{\kt}{\;$\mathscr{B}_t$}
\newcommand{\kr}{$\pd\mathscr{B}$}
%%%%%%%%%%%%%%%%%%%%%%%%%%%%%%%%%%%%%%%%%%%%%%%%%%%%%%%%%
%%%%%%%%%%%%%%%%%%%%%%%%%%%%%%%%%%%%%%%%%%%%%%%%%%%%%%%%%
%%%%%%%%%%%%%%%%%%%%%%%%%%%%%%%%%%%%%%%%%%%%%%%%%%%%%%%%%
\renewcommand{\familydefault}{\sfdefault} % serifenlose Schrift für das gesamte Dokument
\pagestyle{empty}
%%%%%%%%%%%%%%%%%%%%%%%%%%%%%%%%%%%%%%%%%%%%%%%%%%%%%%%%%
%
% los gehts!
%
%%%%%%%%%%%%%%%%%%%%%%%%%%%%%%%%%%%%%%%%%%%%%%%%%%%%%%%%%
\begin{document}
\section*{Kontinuumsmechanik in \LaTeX{}  \textnormal{(nach \textsc{Bertram})}}
\vfill
\subsection*{Schreibweise}
% hier Verzicht auf geradestehende Mathematik! besser nach DIN 1338 da mikrotypografisch korrekt und klarer!
\begin{tabbing}
$a$ 			\hspace{1cm}\=Skalar, Tensor 0. Stufe \; (klein, kursiv, normales Schriftgewicht)\\
$\ve{a}$ 		\>Vektor, Tensor 1. Stufe\; (klein, kursiv, fett) \\
$\tz{A}$ 		\>Dyade, Tensor 2. Stufe \; (groß, kursiv, fett)\\
$\tv{A}$		\>Tetrade, Tensor 4. Stufe  \; (groß, kursiv, fett, serifenlos)
\end{tabbing}

\subsection*{Mengen und Gruppen}
\begin{multicols}{2}
\begin{tabbing}
\lin			\hspace{1cm}\= Lineare Tensoren\\
% funktioniert auch mit $\lin$  usw.
\inv			\> Invertierbare Tensoren\\
\invpl			\> \dots\\
\invmi		\> \dots\\
\unim		\> Unimodulare Tensoren\\
\unimpl		\>\dots\\
\unimmi		\>\dots\\
\orth			\> Orthogonale Tensoren\\
\orthpl		\>\dots\\
\orthmi		\> \dots\\
\iso			\> Iso-Tensoren\\
\isopl			\> \dots \\
\isomi		\> \dots\\
\sym			\> Symmetrische Tensoren\\
\skw			\> Schiefsymmetrische Tensoren\\
\psym		\> Symmetrische Tensoren mit pos. EW
\end{tabbing}
\end{multicols}
\subsection*{Operatoren}
\begin{tabbing}
$\sp$ 		\hspace{1cm}\= Skalarprodukt\\
$\kp$ 		\> Kreuzprodukt\\
$\dyad$ 		\> Dyadisches Produkt\\
$\div$ 		\> Divergenz\\
$\grad$ 		\>Gradient\\
$\rot$ 		\> Rotation\\
$\spur$ 		\> Spur\\
$\d$ 			\> Ableitung\\
$\pd$ 		\> Partielle Ableitung\\
\end{tabbing}

\subsection*{Räume} 
\begin{tabbing}
\R 			\hspace{1cm}\= Raum der reellen Zahlen\\
\V 			\>Vektorraum\\
\E 			\>\textsc{Euklid}ischer Raum\\
\end{tabbing}

\subsection*{Körper} 
\begin{tabbing}
\k			\hspace{1cm}\=Materieller Körper\\
\kr			\>Körperrand\\
\kn			\>Materieller Körper in der Referenzplatzierung\\
\kt			\>Materieller Körper in der Momentanplatzierung\\
\end{tabbing}
\vfill
Magdeburg, November 2013\hfill Marcus Aßmus
%%%%%%%%%%%%%%%%%%%%%%%%%%%%%%%

%%%%%%%%%%%%%%%%%%%%%%%%%%%%%%%
\end{document}
